\documentclass[14pt, a4paper]{extarticle}
\usepackage{GOST}
\usepackage{array}
\usepackage{verbatim}
\usepackage[detect-all]{siunitx}
\usepackage{amsmath}
\usepackage{amssymb}
\usepackage[utf8]{inputenc}
\usepackage{hyperref}

\usepackage{ifthen}


\usepackage{tempora}



\makeatletter
\renewcommand\@biblabel[1]{#1.}
\makeatother

% Для листинга кода:
\usepackage{listings}
\lstset{ %
	language=c++,                 % выбор языка для подсветки (здесь это С)
	basicstyle=\small\sffamily, % размер и начертание шрифта для подсветки кода
	numbers=left,               % где поставить нумерацию строк (слева\справа)
	numberstyle=\tiny,           % размер шрифта для номеров строк
	stepnumber=1,                   % размер шага между двумя номерами строк
	numbersep=5pt,                % как далеко отстоят номера строк от подсвечиваемого кода
	showspaces=false,            % показывать или нет пробелы специальными отступами
	showstringspaces=false,      % показывать или нет пробелы в строках
	showtabs=false,             % показывать или нет табуляцию в строках
	frame=single,              % рисовать рамку вокруг кода
	tabsize=2,                 % размер табуляции по умолчанию равен 2 пробелам
	captionpos=t,              % позиция заголовка вверху [t] или внизу [b] 
	breaklines=true,           % автоматически переносить строки (да\нет)
	breakatwhitespace=false, % переносить строки только если есть пробел
	escapeinside={\#*}{*)}   % если нужно добавить комментарии в коде
}


%для графиков
\usepackage{pgfplots}
\usepackage{filecontents}
\usetikzlibrary{datavisualization}
\usetikzlibrary{datavisualization.formats.functions}
\begin{filecontents}{Std.dat}
	100 8300
	200 41400
	300 176000
	400 490000
	500 960000
\end{filecontents}

\begin{filecontents}{Vin.dat}
	100 4800
	200 34000
	300 140000
	400 391000
	500 760000
\end{filecontents}

\begin{filecontents}{VinMod.dat}
	100 3000
	200 23500
	300 98000
	400 289000
	500 563000
\end{filecontents}


\begin{filecontents}{Std2.dat}
	101 8500
	201 42200
	301 180000
	401 491000
	501 977000
\end{filecontents}

\begin{filecontents}{Vin2.dat}
	101 4850
	201 34500
	301 141000
	401 392000
	501 780000
\end{filecontents}

\begin{filecontents}{VinMod2.dat}
	101 3050
	201 24000
	301 101000
	401 291000
	501 565000
\end{filecontents}


\begin{document}
	
	\begin{table}[ht]
		\centering
		\begin{tabular}{|c|p{400pt}|} 
			\hline
			\begin{tabular}[c]{@{}c@{}} \includegraphics[scale=1]{baum.jpg} \\\end{tabular} &
			\footnotesize\begin{tabular}[c]{@{}c@{}}\textbf{Министерство~науки~и~высшего~образования~Российской~Федерации}\\\textbf{Федеральное~государственное~бюджетное~образовательное~учреждение}\\\textbf{~высшего~образования}\\\textbf{«Московский~государственный~технический~университет}\\\textbf{имени~Н.Э.~Баумана}\\\textbf{(национальный~исследовательский~университет)»}\\\textbf{(МГТУ~им.~Н.Э.~Баумана)}\\\end{tabular}  \\
			\hline
		\end{tabular}
	\end{table}
	\noindent\rule{\textwidth}{4pt}
	\noindent\rule[14pt]{\textwidth}{1pt}
	\hfill 
	\noindent
	\makebox{ФАКУЛЬТЕТ~}%
	\makebox[\textwidth][l]{\underline{~«Информатика и системы управления»~~~~~~~~~~~~~~~~~~~~~~~~~~~~~~~~~}}%
	\\
	\noindent
	\makebox{КАФЕДРА~}%
	\makebox[\textwidth][l]{\underline{~«Программное обеспечение ЭВМ и информационные технологии»~}}%
	\\
	
	\begin{center}
		\vspace{1.5cm}
		{\bf\huge Отчёт\par}
		{\bf\Large по лабораторной работе № 3\par}
		\vspace{0.7cm}
	\end{center}
	
	
	\noindent
	\makebox{\large{\bf Название:}~~~}
	\makebox[\textwidth][l]{\large\underline{~Алгоритмы сортировки~~~~~~~~~~~~~}}\\
	
	\noindent
	\makebox{\large{\bf Дисциплина:}~~~}
	\makebox[\textwidth][l]{\large\underline{~Анализ алгоритмов~~~~~~~~~~~~~~~~~~~~~~~~~~}}\\
	
	\vspace{1.5cm}
	\noindent
	\begin{tabular}{l c c c c c}
		Студент      & ~ИУ7-55Б~               & \hspace{2.5cm} & \hspace{2cm}                 & &  Д.В. 
		Сусликов \\\cline{2-2}\cline{4-4} \cline{6-6} 
		\hspace{3cm} & {\footnotesize(Группа)} &                & {\footnotesize(Подпись, дата)} & & {\footnotesize(И.О. Фамилия)}
	\end{tabular}
	
	\noindent
	\begin{tabular}{l c c c c}
		Преподователь & \hspace{5cm}   & \hspace{2cm}                 & & ~~~~~~Л.Л. Волкова~~~~~~\\\cline{3-3} \cline{5-5} 
		\hspace{3cm}  &                & {\footnotesize(Подпись, дата)} & & {\footnotesize(И.О. Фамилия)}
	\end{tabular}
	
	\vspace{0.6cm}
	\begin{center}	
		\vfill
		\large \textit {Москва, 2020}
	\end{center}
	
	\thispagestyle {empty}
	\pagebreak
	
	% СОДЕРЖАНИЕ 
	\clearpage
	\tableofcontents
	
	
	% ВВЕДЕНИЕ
	\clearpage
	\section*{Введение}
	\addcontentsline{toc}{section}{Введение}
	Цель работы: изучение алгоритмов сортировки массивов. В данной лабораторной работе рассматриваются 3 алгоритма:
	\begin{enumerate}
		\item[1)] сортировка пузырьком;
		\item[2)] сортировка вставками;
		\item[3)] сортировка выбором. 
	\end{enumerate}\par

	В лабораторной работе требуется:
	\begin{enumerate}
		\item[1)] изучить алгоритмы сортировки массивов;
		\item[2)] дать теоритическую оценку алгоритмам сортировки: пузырьком, вставками, выбором;
		\item[4)] реализовать три алгоритма сортировки массивов;
		\item[5)] сравнить алгоритмы сортировок.		
	\end{enumerate}

	\clearpage
	\section{Аналитический раздел}
	В данном разделе представлены математические описания алгоритмов сортировки массивов.
	
	\subsection{Алгоритм сортировки пузырьком}
	Сортировка пузырьком – простейший алгоритм сортировки, применяемый чисто для учебных целей. Практического применения этому алгоритму нет, так как он не эффективен, особенно если необходимо отсортировать массив большого размера. К плюсам сортировки пузырьком относится простота реализации алгоритма.
	
	Алгоритм сортировки пузырьком сводится к повторению проходов по элементам сортируемого массива. Проход выполняет внутренний цикл. За каждый проход сравниваются два соседних элемента, и если порядок неверный элементы меняются местами. Внешний цикл будет работать до тех пор, пока массив не будет отсортирован. Таким образом внешний цикл контролирует количество срабатываний внутреннего цикла Когда при очередном проходе по элементам массива не будет совершено ни одной перестановки, то массив будет считаться отсортированным.%\hyperref[literature]{[1]}
	
	\subsection{Алгоритм сортировки вставками}
	Сортировка вставками — достаточно простой алгоритм. Как в и любом другом алгоритме сортировки, с увеличением размера сортируемого массива увеличивается и время сортировки.
	
	Сортируемый массив можно разделить на две части — отсортированная часть и неотсортированная. В начале сортировки первый элемент массива считается отсортированным, все остальные — не отсортированные. Начиная со второго элемента массива и заканчивая последним, алгоритм вставляет неотсортированный элемент в нужную позицию в отсортированной части. Таким образом, за один шаг сортировки отсортированная часть массива увеличивается на один элемент, а неотсортированная часть уменьшается на один.
	
	\subsection{Алгоритм сортировки выбором}
	Сотрировка выбором - один из простоый алгоритмов. удя по названию сортировки, необходимо что-то выбирать (максимальный или минимальный элементы массива). Алгоритм сортировки выбором находит в исходном массиве максимальный или минимальный элементы, в зависимости от того как необходимо сортировать массив, по возрастанию или по убыванию. Если массив должен быть отсортирован по возрастанию, то из исходного массива необходимо выбирать минимальные элементы. Если же массив необходимо отсортировать по убыванию, то выбирать следует максимальные элементы.
	
	Допустим необходимо отсортировать массив по возрастанию. В исходном массиве находим минимальный элемент, меняем его местами с первым элементом массива. Уже, из всех элементов массива один элемент стоит на своём месте. Теперь будем рассматривать не отсортированную часть массива, то есть все элементы массива, кроме первого. В неотсортированной части массива опять ищем минимальный элемент. Найденный минимальный элемент меняем местами со вторым элементом массива и т. д. Таким образом, суть алгоритма сортировки выбором сводится к многократному поиску минимального (максимального) элементов в неотсортированной части массива.
	
	\subsection*{Вывод}
	\addcontentsline{toc}{subsection}{Вывод}
	Можно сделать вывод, что, помимо приведенных выше алгоритмов сортировки массива, есть множество других не менее или даже более эффективных.
	
	
	\clearpage
	\section{Конструкторский раздел}
	В данном разделе представлены схемы алгоритмов и дана оценка их трудоемкости.
	
	\subsection{Схемы алгоритмов}
	Ниже на Рисунке 1 представлена схема алгоритма сортировки пузырьком.
	\begin{figure}[h!]
		\centering{\includegraphics[scale= 0.8]{bubble_sort.png}}
		\caption*{Рисунок 1 - Схема алгоритма сортировки пузырьком}
	\end{figure}
	
	\clearpage
	Ниже на Рисунке 2 изображена схема алгоритма сортировки вставками.
	\begin{figure}[h!]
		\centering{\includegraphics[scale= 0.8]{ins.png}}
		\caption*{Рисунок 2 - Схема алгоритма сортировки вставками}
	\end{figure}

	\clearpage
	Ниже на Рисунке 3 показана схема алгоритма сортировки выбором.
	\begin{figure}[h!]
		\centering{\includegraphics[scale= 0.8]{ins.png}}
		\caption*{Рисунок 2 - Схема алгоритма сортировки вставками}
	\end{figure}
	
	
\end{document}